%%%%%%%%%%%%%%%%%%%%%%%%%%%%%%%%%%%%%%%%%%%%%%%%%%%%%%%%
% Author: Xinyu Cao                                    %
% B.S.E. Telecommunication Engineering, SDUST           %
%%%%%%%%%%%%%%%%%%%%%%%%%%%%%%%%%%%%%%%%%%%%%%%%%%%%%%%%

\documentclass{resume} % Use the custom resume.cls style

\begin{document}

\introduction[
    fullname=Xinyu Cao,
    email=cao2xinyu@outlook.com,
    linkedin=https://www.linkedin.com/in/xinyu-cao-704727231/,
    github=https://github.com/krisplusmagger
]

%\summary{Senior telecommunication engineering student seeking internship position in the telecommunication industry. Project experience includes applications of software and hardware.  }

\education{
    \educationItem[
        university=Shandong University of Science and Technology,
        college=College of Electronics and Information Engineering,
        graduation=September 2020 - June 2024(Expected),
        grade=84.58/100 GPA,
        program=B.S.E. Telecommunication Engineering,
        coursework=Principles of Communication{,} Information Theory and Coding{,}	Digital Signal Processing{,}	Signals and Systems{,} Computer Network{,} Computer Systems
    ]
}

\skills{
    \skillItem[
        category=Embedded Systems,
        skills=STM32{,}STC15{,} DDS9959{,}UART{,} IIC{,} SPI{,}
    ] \\
    \skillItem[
        category=Design and Modeling Tools,
        skills=MATLAB{,}Proteus{,} Multisim{,} Numpy{,} Pandas, Tensorflow
    ] \\
    \skillItem[
        category=Programming,
        skills=Python{,} C{,} JAVA
    ] \\
    \skillItem[
        category=Technical skills,
        skills=Git{,} Unix
    ] \\
    \skillItem[
        category=English,
        skills=IELTS 6.5{,} CET-6 493{,} CET-4 532
    ] \\
 
}

\begin{workSection}{Research Experience}

    \experienceItem[
        company=TI Cup National Undergraduate Electronics Design Contest,
        location=Jinan,
        position=Signal Separation Device,
        duration=August 2023
    ]
    \begin{itemize}
        \itemsep -6pt
        \item Designed and implemented a signal separation device incorporating FFT frequency domain measurement, Software Phase-Locked Loop, and Operational Amplifier Circuit Design.
        \item Developed a unity gain adder using OPA1612, encompassing schematic and PCB design. Conducted circuit simulations using Multisim, employed STM32F103ZE for ADC acquisition, FFT computation, and waveform identification. Achieved signal regeneration using AD9959 and AD9934. Employed a software phase-locked loop approach, incorporating AD620 and low-pass filters to maintain co-frequency and co-phase alignment between original and regenerated signals.
    \end{itemize}
    
    \experienceItem[
        company=Lichuang Cup SDUST Electronic Design Competition,
        location=Qingdao,
        position=Basic Oscilloscope and Signal Generator,
        duration=April 2023
    ]
    \begin{itemize}
        \itemsep -6pt
        \item Developed a basic oscilloscope using STM32 microcontroller, involving ADC, DAC, Direct Memory Access, and waveform recognition. Implemented sampling, quantization, and FFT for waveform analysis.
    \end{itemize}

\end{workSection}
\begin{workSection}{Academic Projects}

    \customItem[
        title=DTMF System Simulation and Implementation,
        duration=May 2023 - June 2023,
        keyHighlight=Developed an innovative approach using ANN model for DTMF signal demodulation.
    ]
    \begin{itemize}
        \itemsep -6pt
        \item Implemented a real-time DTMF system using Python, TensorFlow, and STM32 microcontrollers.
        \item Utilized two STM32 boards for building and testing the system's performance.
        \item Designed a Deep Neural Network (DNN) model with TensorFlow, showcasing exceptional performance, especially in challenging low signal-to-noise ratio conditions.
    \end{itemize}

    \customItem[
        title=Modulation Recognition using Conventional Neural Networks,
        duration=Spring 2023,
        keyHighlight=Explored modulation recognition with deep learning aapproaches.
    ]
    \begin{itemize}
        \itemsep -6pt
        \item Analyzed signal samples under low Signal-to-Noise Ratio (SNR) to classify modulation types (Python, Numpy).
        \item Developed and evaluated DNN, CNN, and CLDNN models on RadioML datasets.
        \item Demonstrated the effectiveness of blind Convolutional Networks in classifying time series radio signal data.
    \end{itemize}

    \customItem[
        title=Digital Speech Signal Analysis and Recognition,
        duration=December 2022 - January 2023,
        keyHighlight=Applied audio processing techniques for MFCC-based speech signal analysis.
    ]
    \begin{itemize}
        \itemsep -6pt
        \item Employed FSDD dataset and audio processing techniques (e.g., speech framing, Mel filters, MFCC, Gammatone cepstral coefficients, KNN) to extract feature vectors from 0 to 9-digit speech signals.
        \item Utilized MATLAB for data analysis and signal processing to achieve accurate recognition and classification.
    \end{itemize} 
  
\end{workSection}
\end{document}
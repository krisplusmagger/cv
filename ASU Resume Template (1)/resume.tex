%%%%%%%%%%%%%%%%%%%%%%%%%%%%%%%%%%%%%%%%%%%%%%%%%%%%%%%%
% Author: Vignesh Iyer                                 %
% MS CSE ASU                                           %
%%%%%%%%%%%%%%%%%%%%%%%%%%%%%%%%%%%%%%%%%%%%%%%%%%%%%%%%

\documentclass{resume} % Use the custom resume.cls style

\begin{document}

\introduction[
    fullname=Xinyu cao,
    email=cao2xinyu@outlook.com,
    phone=,
    linkedin=https://www.linkedin.com/in/xinyu-cao-704727231/,
    github=https://github.com/krisplusmagger
]

%\summary{Senior telecommunication engineering student seeking internship position in the telecommunication industry. Project experience includes applications of software and hardware.  }

\education{
    \educationItem[
        university=Shandong University of Science and Technology{} {} ,
        college=College of Electronic and Information Engineering{},
        ,
        graduation=September 2020 - June 2024(Expected),
        grade=84.58/100 GPA,
        program=B.S.E.{ } Telecommunication Engineering,
        coursework=Principles of Communication{,} Information Theory and Coding{,} Digital Signal Processing{,} Signals and Systems{,} Computer Network{,} Computer Systems{}     ]
}

\skills{
    \skillItem[
        category=Embedded Systems,
        skills=STM32{,} STC15{,} DDS9959{,}	UART{,}	IIC{,} Proteus
    ] \\
    \skillItem[
        category=Design and Modeling Tools,
        skills=MATLAB{,} Numpy{,} Pandas{,} Tensorflow
    ] \\
    \skillItem[
        category=Programming,
        skills=Python{,} C{,} JAVA
    ] \\
    \skillItem[
        category=Technical skills,
        skills=Git{,} Unix
    ] \\
      \skillItem[
        category=English,
        skills=IELTS 6.5{,}  CET-6 493{,}  CET-4 532
    ] \\
 
}

\begin{workSection}{Experience}

     
    \experienceItem[
        company= TI Cup National Undergraduate Electronics Design Contest,
        location=Jinan{} ,
        position=Signal Separation Device,
        duration=August 2023 
    ]
    \begin{itemize}
        \itemsep -6pt {} 
        \item FFT frequency domain measurement{,} Software Phase-Locked Loop{,} Operational Amplifier Circuit Design
        \item Implemented a unity gain adder using OPA1612, encompassing schematic and PCB design. Conducted circuit simulations using Multisim, employed STM32F103ZE for ADC acquisition, FFT computation, and waveform identification. Achieved signal regeneration using AD9959 and AD9934. Employed a software phase-locked loop approach, incorporating AD620 and low-pass filters to maintain co-frequency and co-phase alignment between original and regenerated signals.

       
     \end{itemize}
     
      \experienceItem[
        company= Lichuang Cup SDUST Electronic Design  Competition,
        location=Qingdao{} ,
        position= Basic oscilloscope and signal Generator,
        duration=April 2023 
    ]
    \begin{itemize}
        \itemsep -6pt {} 
        \item ADC, DAC, Direct Memory Access{,} Waveform recognition 
        \item Developed a basic oscilloscope using STM32 microcontroller Implemented sampling, and quantization, and utilized Fast Fourier Transform (FFT) for waveform analysis. Expertise in signal processing and troubleshooting.
       
     \end{itemize}
     
\end{workSection}

\begin{workSection}{Academic projects}
     \customItem[
        title=DTMF System Simulation and Implementation
        ,
        duration=May 2023 - June 2023,
        keyHighlight=Developed a new approach using ANN model to demodulate DTMF signals.
     ]
     \begin{itemize}
        \vspace{-0.5em}
        \itemsep -6pt {} 
        \item This course design was based on Python, TensorFlow, and STM32 and implemented a DTMF system.
        \item Using two STM32 boards to build real-time DTMF systems.
        \item A DNN network was designed using TensorFlow, and its excellent performance was demonstrated compared to other recognition approaches, particularly in low signal-to-noise ratio conditions.
     \end{itemize}

     \customItem[
        title=Modulation recognition using Conventional neural network,
        duration=Spring 2023,
        keyHighlight= Tensorflow{,} DNN{,} CNN{,} CLDNN
     ]
     \begin{itemize}
        \vspace{-0.5em}
        \itemsep -6pt {} 
        \item Assessed signal samples datasets under low SNR to determine the possible types of modulation (Python{,} Numpy).
        \item There are three proposed architectures - DNN{,} CNN{,} CLDNN which are trained and evaluated on RadioML Datasets.
        \item Proving blind Convolutional Networks
        on time series radio signal data are viable and work quite well.
     \end{itemize}

    \customItem[
        title= Digital Speech Signal Analysis and Recognition,
        duration=December 2022 - January 2023,
        keyHighlight=MFCC{,} MRMR{,} K-Nearest Neighbor{,}
        MATLAB
     ]
     
     \begin{itemize}
        \vspace{-0.5em}
        \itemsep -6pt {} 
        \item In this design, the FSDD dataset(similar to MNIST dataset but in audio form) and various audio processing techniques were employed to extract feature vectors from 0 to 9-digit speech signals. These techniques include speech framing with windowing{,} Mel filters{,} MFCC{,} Gammatone cepstral coefficients{,} KNN.

     \end{itemize} 
  
\end{workSection}

\end{document}